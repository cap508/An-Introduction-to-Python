\section{Introduction} \label{sec:WhyPython}


In 1981 I received my first computer, a BBC Model B microcomputer. It might seem difficult to believe now but this was a big deal. Computers were the future, they were exciting and the possibilities seemed endless. I plugged the machine into my TV and switched it I was rewarded with a couple of beeps and a blinking cursor waiting for me to type on the keyboard. 

\begin{verbatim}
   BBC Computer 32K
   
   BASIC
   
   > \_
\end{verbatim}


This was the beginning of my programming adventure.

It might be over 40 years since that first computer, but for me, the joy I get from programming is still there. Learning to program that provided me with opportunities I couldn't have imagined and today the possibilities look bigger and more impressive than ever.

What I want to do through this short course is remove some of the mystery surrounding  programming and show you that with very little effort your laptop can be so much more than a portable video player. I want you to take control of your computer and understand how easy and useful computer programming can be.


\subsection{Who is this book aimed at?}

This book is being written in the first instance to support post graduate students from non-scientific disciplines who are undertaking a multi-disciplinary PhD in Safe Artificial Intelligence. I assume that you have no previous programming experience, no experience of data analysis and no experience of building models using data.

By the end of this book however I expect you to be able to write computer program to do all of these things. You will be able to create graphs for use in your thesis and you will even build and analyse a machine learning models to predict system failure and even turn pictures into text.


\subsection{So what are we going to learn?}

Through this course you are going to learn how to program in Python. This isn't because I believe Python to be a wonderful language but because I believe it's a useful language in this context. It will allow us to use most of the commonly used programming structures and paradigms and it is widely supported in data intensive applications, such as artificial intelligence.

In the first part of the book we are going to concentrate on the basics of interacting with python. We will start by looking at how we interact with python in an interpretive environment before moving onto using a dedicated programming environment (VSCode) to write and debug our code.

We will then look at how we use python to store information introducing concepts like variables, data types, list and dictionaries before moving onto programming structures which allow us to make choices and create programs that can do the same thing many times.

In the second part of the book we will look at how we can gather all the fragments of code we understand from part one into blocks which can be reused. We will consider how we can build complex programs from simple blocks.  We'll start our exploration of program structures by looking at functional programming, before moving onto, the more advanced, object orientated programming. 


\if 0
Tuesday PM : Functional and OO Programming
Functions
Libraries
Debugging
Arguments
File Handling
Classes

Wednesday AM : Managing Data
?Arrays structures : numpy
Data Handling : Pandas
Reading CSVs
Data Frames
Data Filtering
Data Imputation

Wednesday PM : Making Graphs 
Line, Bar, Scatter
Multiple charts
Managing Axes
Annotation

Thursday : A Machine Learning Project
Machine Learning Libraries
Building Models
Decision trees
Neural Networks
Evaluating Models
Metrics
Confusion Matrices

\fi


\subsection{What do I need?}


I have t decide how much detail to go into here aas the bootcamp shouldn't need them to install anything!.


Laptop, Python 3, VSCode, Anaconda Navigator.